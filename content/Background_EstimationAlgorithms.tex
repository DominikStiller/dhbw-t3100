Obtaining the true state of a physical system is next to impossible. Any sensor measurement contains noise and other errors, which distort our observations of the underlying process. Therefore, the challenge is to minimize the effect of these errors. This is known as \textit{filtering problem}, where we want to obtain the best estimate $\hat{x}_k$ of the system state $x_k$ at time step $k$ based on past observations $z_i, i \leq k$~\cite[p.~67]{Mitter.1996}. Usually, this involves the fusion of multiple sensors and physical models. In this section we explore common estimation methods to solve the filtering problem. We will regard their discrete-time versions instead of their continuous-time versions, since digital observations are based on sampling with a finite rate.

\subsection{Extended Kalman Filter}\label{sec:background-ekf}
The Kalman filter~\cite{Kalman.1960} is the most widely used estimation algorithm~\cite[p.~401]{Julier.2004}. With the trajectory estimation for the Apollo space program being one of its first notable applications~\cite{Grewal.2010}, it can now be found not only in many vehicles, aircraft, spacecraft, but even in finance and econometrics. The Kalman filter assumes a linear system, which is often not the case or only true in a small region. Therefore, a non-linear version based on the same ideas has been developed with the \gls{ekf}, which we will regard in this section.

The idea of the \gls{ekf} is simple yet powerful: we observe the physical system through noisy measurements but simultaneously predict the state with a mathematical model of the system. At every time step, we fuse the observations with the predictions based on our confidence in either, effectively resulting in a weighted average. For example, a noisy measurement is rather unreliable, while models can never capture all nuances and uncertainties, both of which is reflected in the weighting. The fusion yields a good estimate of the true state of the system, better than only measurements or predictions could alone. Note that, while the Kalman filter is the optimal estimator under some conditions~\cite{Maybeck.1991}, the \gls{ekf} is generally not optimal for non-linear systems.

The model describes the \textit{state} $x_k$ of the system at time $k$. For example, a train on a track can be described by its position and velocity which are related, so the state is two-dimensional. How the state evolves over time according to a process is shown in the \textit{process equation} \ref{eq:ekf-process-equation}. The process function $f(x_{k-1}, u_k)$ propagates the previous state $x_{k-1}$ and is controlled by external inputs $u_k$ such as the train engine accelerating the train, and disturbed by process noise $w_k$, which represents unmodeled behavior like air resistance or friction. The process noise is assumed to be Gaussian with a zero mean and a diagonal covariance matrix $Q_k$, i.e no noise terms are correlated.
\begin{equation}\label{eq:ekf-process-equation}%
x_k = f(x_{k-1}, u_k) + w_k%
\end{equation}

The state is inherent to the system, but cannot be directly observed. Rather, we have \textit{observations} $z_k$. The elements of $z_k$ and $x_k$ must not necessarily coincide, i.e. the observation space and state space can differ; for example, we might have multiple sensors for the train's velocity but none for the position. The measurement by which we obtain the observations is described by the \textit{measurement equation} \ref{eq:ekf-measurement-equation}. The measurement function $h(x_k)$ gives the observation vector which is disturbed by measurement noise. The measurement noise is assumed to be Gaussian with a zero mean and a diagonal covariance matrix $R_k$. We use the measurement equation to predict measurements from our model state that we can fuse with real measurements.
\begin{equation}\label{eq:ekf-measurement-equation}%
z_k = h(x_k) + v_k%
\end{equation}

The process and measurements equations acknowledge that the state can only be modeled and observed with some degree of uncertainty. Therefore, in addition to the state estimate $\hat{x}_k$, we want to keep track of its covariance $P_k$, which is zero mean as well. Together with linearizations, the zero mean assumption allows the \gls{ekf} to be much more computationally efficient than more generally applicable algorithms like the particle filter.

The \gls{ekf} algorithm~\cite[p.~16~ff.]{Haykin.2001} is recursive, with each iteration comprising a prediction step and a correction step, i.e. the previous state is first propagated to the current time step using the model and then corrected with the measurements from the real system. In the following description of the algorithm, we distinguish between the predicted state estimate $\hat{x}_k^-$ and covariance $P_k^-$, and the corrected state estimate $\hat{x}_k^+$ and covariance $P_k^+$.

\begin{description}
\item[Prediction] At the beginning of every iteration, the corrected state from the previous iteration is propagated using the process function to predict the current state (equation \ref{eq:ekf-state-propagation}). Note that the prediction only depends on the directly previous state, not any other earlier states in the history, manifesting that the \gls{ekf} assumes an underlying Markov process.
\begin{equation}\label{eq:ekf-state-propagation}%
\hat{x}_k^- = f(\hat{x}_{k-1}^+, u_k)%
\end{equation}

The previous covariance also needs to be propagated, since the estimate uncertainty might change with every step (equation \ref{eq:ekf-cov-propagation}). The Jacobian matrix $F_k$ of the process function, which can be though of as a vector derivative, needs to be evaluated for this step (equation \ref{eq:ekf-process-jacobian}). Additionally, the process noise is considered to account for a model mismatch, since there are always aspects of the real-world which cannot be modeled. Therefore, leveraging the additive property of Gaussian noise, the process covariance $Q_k$ is added to the propagated covariance.
\begin{align}\label{eq:ekf-cov-propagation}%
P_k^- &= F_k P_{k-1} F_k^T + Q_k \\%
\label{eq:ekf-process-jacobian}%
F_k &= \left. \frac{\partial f}{\partial x} \right|_{x = \hat{x}_{k-1}^+}%
\end{align}

The measurement function then predicts the expected measurements from the estimated state (equation \ref{eq:ekf-measurement-prediction}).
\begin{equation}\label{eq:ekf-measurement-prediction}%
\hat{z}_k = h(\hat{x}_k^-)%
\end{equation}


\item[Correction] Once the prediction has been made, it can be corrected with observations obtained from the real system by taking a weighted average of both. The weighting is determined by the \textit{Kalman gain} $K_k \in [0, 1]$ (equation \ref{eq:ekf-kalman-gain}), which also maps back from the observation space into the state space using the Jacobian matrix $H_k$ of the measurement function (equation \ref{eq:ekf-measurement-jacobian}). The Kalman gain is defined as the ratio of the state-innovation covariance $P_{xy}$ to the innovation covariance $P_{yy}$, and so it captures the uncertainty in the measurements and the model prediction. When the measurement covariance $R_k$ is small relative to the model covariance, the Kalman gain is high and so the measurement is weighted more. On the other hand, when the measurement noise is high, the Kalman gain approaches zero and the model prediction is weighted more.
\begin{align}\label{eq:ekf-kalman-gain}%
K_k &= \underbrace{P_k^- H_k^T}_\textrm{$P_{xy}$} {\underbrace{(H_k P_k^- H_k^T + R_k)}_\textrm{$P_{yy}$}}^{-1} \\%
\label{eq:ekf-measurement-jacobian}%
H_k &= \left. \frac{\partial h}{\partial x} \right|_{x = \hat{x}_{k}^-}%
\end{align}

In the best case, the model and the measurements coincide. However, this is usually not the case, so the real measurements $z_k$ and the predicted measurements $\hat{z}_k$ differ, resulting in an \textit{innovation} $y_k$, also called \textit{residual}. The absolute innovation is higher when there is a larger mismatch between the model and the real system. The Kalman gain now determines, how much of the residual is used to correct the prediction. The posterior state estimate $\hat{x}_k^+$ is the final, best estimate for the iteration $k$ and will be used in the next time step (equation \ref{eq:ekf-state-correction}).
\begin{equation}\label{eq:ekf-state-correction}%
\hat{x}_k^+ = \hat{x}_k^- + K(\underbrace{z_k - \hat{z}_k}_\textrm{$y_k$})%
\end{equation}

Additionally, the covariance matrix needs to be updated to account for the correction (equation \ref{eq:ekf-cov-correction}, $I$ being the identity matrix). The covariance is reduced based on the Kalman gain, signifying how the confidence in our estimate has increased by the correction.
\begin{equation}\label{eq:ekf-cov-correction}%
P_k^+ = (I - K H_k)P_k^-%
\end{equation}
\end{description}


\subsection{Unscented Kalman Filter}
While the \gls{ekf} has been successfully applied to problems for a long time, it suffers from a number of limitations~\cite[p.~402]{Julier.2004}:
\begin{itemize}
\item The error propagation in equation \ref{eq:ekf-cov-propagation} is only reliably if the linearization in form of the Jacobian matrix $F_k$ is sufficiently accurate. This might not be the case in highly non-linear systems or under the wrong initial estimate. In the worst case, the estimate might even diverge.
\item Linearization can only be applied to differentiable functions, i.e. processes for which a Jacobian exists. Discontinuities and singularities might exist in some cases, making differentiation impossible.
\item The derivation of Jacobian matrices can be very difficult and error-prone for complex systems, both the equations themselves and the conversion to code.
\end{itemize}
Therefore, alternatives to the \gls{ekf} have been developed. A promising algorithm is the \gls{ukf}~\cite{SimonJ.Julier.1997}.

At its core, the \gls{ukf}'s improvement relies on the \gls{ut}. This transformation is a method to approximate the properties of a statistical distribution undergoing a non-linear transformation. A small set of sigma points ($2n+1$ for an $n$-dimensional random variable) is sampled from the distribution using a specific, deterministic algorithm (not randomly as in Monte Carlo algorithms). These points contain second-order information about the distribution and can be transformed to capture properties such as mean and variance of the resulting distribution. Therefore, the mean is calculated to a higher order of accuracy than in an \gls{ekf}, whereas the covariance is calculated to the same order of accuracy. Through the \gls{ut}, linearization of the Jacobians is rendered unnecessary, which aids the filter's numerical stability in non-linear systems and enables application in discontinuous systems. Furthermore, no manual derivation of the Jacobian is required. Still, the efficiency and accuracy is equal or better than the \gls{ekf}.

The \gls{ukf} leverages the \gls{ut} to adjust the estimate and covariance in both the prediction and correction step. While the general structure of the Kalman filter including prediction and correction remains, the sigma points are now used in every step.

The \gls{ukf} has been shown to outperform the \gls{ekf} in terms of accuracy in a large number of applications. While it too struggles with second-order moments, only the second-order \gls{ekf}~\cite{M.Roth.2011} can match its accuracy, but is more computationally complex~\cite{Gustafsson.2012}. Therefore, the \gls{ekf}'s popularity over the objectively better \gls{ukf} can probably only be attributed to its long existence.
