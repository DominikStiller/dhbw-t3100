In this thesis, we first presented different approaches for state estimation and failure detection. Then we described the design and implementation of a robust and flexible state estimation for a race car. At its core is a simple three-stage architecture, comprising a preprocessing component for measurement harmonization and \gls{imu} fusion, a unified failure and sensor setup detection providing robustness and support for flexible sensor sets, and an \gls{ekf}-based state estimation performing model-based sensor fusion. Our evaluation shows that the provided state estimate is accurate even in the presence of outliers and sensor failures, but works best when all sensors are available.

While the estimation of the vehicle velocity is already good, it can likely be improved with a better algorithm for transformation of the motor speeds such as the ones presented in~\cite{Song.2002}. Automated calibration of the \glspl{imu}, i.e. determining the orientation so measurements can be aligned with the vehicle axes, possibly every time the state estimation is started, could be beneficial as well since many components rely on good \gls{imu} measurements. Lastly, a tighter integration with the \gls{dv} software and camera/lidar information could improve state estimation and vehicle performance in the future.
